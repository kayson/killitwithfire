\section{Discussion}
Our implementation of the simulation and rendering achieves a detailed and visually plausable fire. Physically based variables can be fine tuned to achieve a wide variety of flame types by adjusting fuel injection speed, fuel ignition temperature, flame speed, fuel density, cooling constants etc.  Using world coordinates for our simulation grids enabled us to have different resolutions for different grids used in the simulation. For instance the temperature grid used for rendering can have considerably higher resolution than the grids used for the velocity fields in the same simulation run.

The design patterns used in our implementation is based on class inheritance, which turns out to be considerable performace hit, and created a sometimes unnecessary long simulation time. The current render method only samples from the simulation grids and does not perform any intersection tests against meshes or any triangles for that matter. We can therefore currently not render triangles, meshes, and so on or anything not present as data in the grids. The implemented calculation of the radiance values for the surrounding area is a quick method, but costly in memory since the radiance is calculated for every wavelength in every voxel, which is needed in order to calculate the mean radiance value. A more correct method would probable implement a Monte Carlo method instead.
In order to completely solve the Navier-Stokes equations, an equation on the following from must first be worked out:
\begin{equation}\label{eq:advect}\frac{\partial q}{\partial t} +\nabla q \cdot \vec{u}= 0\end{equation}
Where \begin{math}q= q(t,\vec{x})\end{math} being some quantity, moving with the velocity field of the fluid, \begin{math}\vec{u}\end{math}, at a certain time \begin{math}t\end{math} and point \begin{math}\vec{x}\end{math} in space. L.h.s in eq.\ref{eq:advect} is is also what defines the material derivative, denoted using capital D:
\begin{equation}\frac{Dq}{Dt} = \frac{\partial q}{\partial t} +\nabla q \cdot \vec{u}\end{equation}
The material derivative describes the change of a quantity in a velocity field. An equation using the material derivative is called an advection equation and since the material being advected is the velocity field itself, \begin{math}q = \vec{u}(t,\vec{x})\end{math} eq. \ref{eq:self-advect} is often referred to as Self Advection.
\begin{equation}\label{eq:self-advect}\frac{\partial u}{\partial t} +\nabla u \cdot \vec{u}= 0\end{equation}
From the Eulerian perspective, which is main focus in this report, we are observing fixed points in space and tracing the change of velocity at these points while the Lagrangian point of view is focusing on a fixed set of particles, and tracing their trajectory (position). Equation \ref{eq:self-advect} states that the velocity is only changing at a location due to movement/replacement of quantity, or more easily explained in Lagrangian terms, the particles are not changing their velocity to any external force, just moving with the flow. The Lagrangian reasoning is fundamental to a commonly used method to find a numerical solution to eq. \ref{eq:self-advect} as described in more detail in section \ref{sec:self-advection-implementation}.

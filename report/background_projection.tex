The last step in solving the incompressible Navier-Stokes equation is the pressure step, which is also sometimes called the projection step, by its linear algebraic properties of a projection. This step enforces that the velocity field is divergence free, and therefore incompressible, by calculating a pressure field \emph{p} which satisfies this property, eq. \ref{eq:navierstokes}. 

This can be done by using the Helmholtz-Hodge decomposition theorem which states that a vector field $\vec{v}$ can be expressed as a sum of two vector fields, one curl free $\vec{v_{cf}}$ and one divergence free $\vec{v_{df}}$, eq. \ref{eq:helm_holtz_1}. If $\vec{v}$ is then set as $\vec{v_{ext}}$ (the velocity field given by the external forces step) and $\vec{v_{df}}$ as the requested velocity field $\vec{v_{req}}$ and then set $\vec{v_{cf}}$ as $\Delta t \frac{\nabla \emph{p}}{\rho}$, it is possible to reorder eq. \ref{eq:helm_holtz_1} to eq. \ref{eq:helm_holtz_2} which looks like a euler integration.

\begin{subequations}
\label{eq:helm_holtz}
\begin{equation}
\label{eq:helm_holtz_1}
\vec{v} = \vec{v_{cf}} + \vec{v_{df}}
\end{equation}    
\begin{equation}
\label{eq:helm_holtz_2}
\vec{v_{ext}} = \Delta t \frac{\nabla \emph{p}}{\rho} + \vec{v_{req}} \Leftrightarrow \vec{v_{req}} = \vec{v_{ext}} - \Delta t \frac{\nabla \emph{p}}{\rho}
\end{equation}
\end{subequations}

It is still two unkowns in eq. \ref{eq:helm_holtz_2} which is one to many to be able to solve this equation. But since $\vec{v_{req}}$ is divergence free it is possible to rule it out by apply the divergence operator $\nabla$ on both sides in eq. \ref{eq:helm_holtz_2}. This results in a possion equation \ref{eq:poission}, which is possible to solve. 

\begin{equation}
\label{eq:poission}
\nabla\vec{v_{ext}} = \Delta t \frac{\Delta \emph{p}}{\rho}
\end{equation}

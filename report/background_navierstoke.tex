The incompressible Navier-Stokes equations in equation \ref{eq:navierstokes} offer a way to describe the motion of fluids. 
This equation is the base of most simulations of fluids eg. water, smoke, and fire. 
The equation consists of a set of partial differential equations that describe how the fluid should behave through out the simulation.
\begin{equation}
  \label{eq:navierstokes}
  \begin{split} 
    \frac{\partial \vec{u}}{\partial t} + \vec{u} \cdot \nabla \vec{u} + \frac{1}{\rho} \nabla p &= \vec{g} + \nu \nabla \cdot \nabla \vec{u},\\ 
    \nabla\cdot \vec{u} &= 0
  \end{split}
\end{equation}
In this equation, $\vec{u}$ represents the velocity field of the fluid. 
$\rho$ is the density of the fluid. 
$p$ stands for the pressure and is the force per unit area that the fluid is affecting its surroundings with. 
The letter $g$ represents external or body forces that acts on the whole fluid. 
This term handles things like gravity and buoyancy. 
%The letter $\rho$ represents the fluids viscosity. 
The term $\nu \nabla \cdot \nabla \vec{u}$ represents the fluild viscosity.
This term describes how much the fluid resists deforming while it is moving. 
Things like honey has a high viscosity, while fluids like water has low viscosity. 
For fluids with low viscosity, this term is usually not modeled because it has such small impact on the simulation.
\\
\\
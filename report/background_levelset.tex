To be able to track the interface between the blue core and the hot gaseous products a Level set method is used. The Level set method represents the interface by an implicit function $\phi$ defined as Eq \ref{eq:levelsetmethod} in a 3D-space. 

\begin{subequations}
\label{eq:levelsetmethod}
\begin{equation}
 L = \left \{ \vec{x} \in \Re^3 : \phi(\vec{x}) = h  \right \}
\end{equation}    
\begin{equation}
  L_{inside} = \left \{ \vec{x} \in \Re^3 : \phi(\vec{x}) \geq h  \right \}
\end{equation}
\begin{equation}
  L_{outside} = \left \{ \vec{x} \in \Re^3 : \phi(\vec{x}) <  h  \right \}
\end{equation}
\end{subequations}

This representation is useful for tracking and calculating topology properties of a interface. The representation makes it easy to find out where in the fluid a sample point is taken since it is only a matter of testing the sign. Since $\phi$ is a implicit function, the normal can be calculated as in Eq \ref{eq:normal1}, and is pointing in an inwards direction. (TODO har jag r�tt h�r?)

\begin{equation}
\label{eq:normal1}
 \vec{n} = \frac{ \nabla \phi }{ \left | \nabla \phi \right | }
\end{equation}    

Moving an interface along a vector field $\vec{V}$, the velocity field for instance, is a matter of solving the Hyperbolic PDE in equation (\ref{eq:levelset_advection}).

 \begin{equation}
\label{eq:levelset_advection}
  \frac{\partial \phi}{\partial t} = -\vec{V} \cdot \nabla \phi
\end{equation}  

$\phi$ is also a signed distance function which means that the value of $\phi$ gives the distance to the closest point $\vec{p}$ of the interface, eq. (\ref{eq:signed_distance}). The direction of this point is also parallel with the normal, and since the length of the gradient is $1$ [K�lla bridson], this point $\vec{p}$ can be calculated for a given point $\vec{x}$ by equation (\ref{eq:find_point}).

\begin{subequations}
\label{eq:signed_distance}
\begin{equation}
distance_L(\vec{x}) = \underset{\vec{p} \in L}{\texttt{min}} \left \| \vec{x} - \vec{p} \right \|
\end{equation}    
\begin{equation}
\phi(\vec{x}) = distance_L(\vec{x}) : \vec{x} is inside
\end{equation}
\begin{equation}
\phi(\vec{x}) = -distance_L(\vec{x}) : \vec{x} is outside
\end{equation}
\end{subequations}

\begin{equation}
\label{eq:normal2}
 \vec{n} = \nabla \phi
\end{equation} 

\begin{equation}
\label{eq:find_point}
 \vec{p} = \vec{x} - \vec{n} \cdot \nabla\phi(\vec{x}), \vec{p} \in L
\end{equation} 

The reason for having a signed distance function is to increase stability, [k�lla lvlset]. 